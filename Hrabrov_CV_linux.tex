%%%%%%%%%%%%%%%%%%%%%%%%%%%%%%%%%%%%%%%%%
% Twenty Seconds Resume/CV
% LaTeX Template
% Version 1.1 (8/1/17)
%
% This template has been downloaded from:
% http://www.LaTeXTemplates.com
%
% Original author:
% Carmine Spagnuolo (cspagnuolo@unisa.it) with major modifications by 
% Vel (vel@LaTeXTemplates.com)
%
% License:
% The MIT License (see included LICENSE file)
%
%%%%%%%%%%%%%%%%%%%%%%%%%%%%%%%%%%%%%%%%%

%----------------------------------------------------------------------------------------
%	PACKAGES AND OTHER DOCUMENT CONFIGURATIONS
%----------------------------------------------------------------------------------------

\documentclass[letterpaper]{twentysecondcv} % a4paper for A4
%----------------------------------------------------------------------------------------
%	 PERSONAL INFORMATION
%----------------------------------------------------------------------------------------

% If you don't need one or more of the below, just remove the content leaving the command, e.g. \cvnumberphone{}

\profilepic{DeXP.jpg} % Profile picture

\cvname{Dmitry Hrabrov} % Your name
\cvjobtitle{Indie game developer} % Job title/career

\cvdate{5 October 1988} % Date of birth
\cvaddress{Belarus, Gomel} % Short address/location, use \newline if more than 1 line is required
\cvnumberphone{+375447606986} % Phone number
\cvsite{http://dexperix.net} % Personal website
\cvskype{dexperix}
\cvmail{soft@dexp.in} % Email address

%----------------------------------------------------------------------------------------

\begin{document}
%----------------------------------------------------------------------------------------
%	 ABOUT ME
%----------------------------------------------------------------------------------------

\aboutme{I've been in the game development since 2014. The first game was made on the RenPy engine (Python language). Others were made in the pure C. I totally love C language and have been using it more than 10 years. The main skill is making crossplatform applications. My personal dream is to make fun computer games that people will like.}


%----------------------------------------------------------------------------------------
%	 SKILLS
%----------------------------------------------------------------------------------------

% Skill bar section, each skill must have a value between 0 an 6 (float)
\skills{%{GIMP (Photoshop)/1.5/4},
{WinAPI/3.3/5},{Android/2.1/2},{JavaScript/2.8/5},{C\#/1.5/1},{C++/1.9/4},{Python/2.5/2},{Java/3.2/4},{Linux/4.1/10},{C/5.6/10}}


%----------------------------------------------------------------------------------------

\makeprofile % Print the sidebar

\section{Education}

\begin{twenty} % Environment for a list with descriptions
	\twentyitem{2012-2016}{Post-Graduation Course, IT}{\href{https://www.gstu.by/}{Gomel State Technical University} }{Topics: Pseudorandom sequences generators; local Wi-Fi positioning}
	\twentyitemmiddle{2011-2012}{Master of Engineering Science, IT}{\href{https://www.gstu.by/}{Gomel State Technical University} } 
	\twentyitemmiddle{2006-2011}{Bachelor of Science, IT}{\href{https://www.gstu.by/}{Gomel State Technical University} } 
	\twentyitem{2004-2006}{Lyceum Graduate}{\href{http://gsrl.by/}{Gomel state regional lyceum} }{Specializing in mathematics and informatics, Olympiad programming}
	%\twentyitem{<dates>}{<title>}{<location>}{<description>}
\end{twenty}

%----------------------------------------------------------------------------------------
%	 PUBLICATIONS
%----------------------------------------------------------------------------------------

\section{Scientific Publications}

More then 40 scientific publications in Russian. Including: 5 articles in scientific magazines, 4 patents, participation in competitions, many conference materials.

%----------------------------------------------------------------------------------------
%	 AWARDS
%----------------------------------------------------------------------------------------

%\section{Awards}

%\begin{twentyshort} % Environment for a short list with no descriptions
%	\twentyitemshort{1987}{All-Time Best Fantasy Novel.}
%	\twentyitemshort{1998}{All-Time Best Fantasy Novel before 1990.}
	%\twentyitemshort{<dates>}{<title/description>}
%\end{twentyshort}



%----------------------------------------------------------------------------------------
%	 EXPERIENCE
%----------------------------------------------------------------------------------------

\section{Code samples / Tools}

\begin{twenty}
	\twentyitem{2017}{dxTarRead}{\href{https://dexp.in/tools/dxtarread/}{https://dexp.in/tools/dxtarread/}}{
A minimalistic non compressed archive file readers written in ANSI C. Supported formats: GNU tar, PAX, GNU ar, Cpio (4 variants).}

\twentyitem{2015}{dxPmdxConverter}{\href{https://github.com/DeXP/dxPmdxConverter}{https://github.com/DeXP/dxPmdxConverter}}{
Simple PMD/PMX to MQO/OBJ converter written in ANSI C. Console version exists for: Linux, Mac OS X, Windows; both 32 and 64 bit.}
\end{twenty}



\section{Open source contributions}

\begin{twenty}
	\twentyitem{2016-2017}{Nuklear}{\href{https://github.com/vurtun/nuklear/commits?author=DeXP}{https://github.com/vurtun/nuklear/commits?author=DeXP}}{
A single-header ANSI C gui library. My changes are mostly related to examples: adding new ones, adding new functionality to existent. 
%Some improvements: adding ability to load images to GDI+ and X11 backend, adding support of OpenGL ES 2.0 and Emscripten.

Used technologies: ansi C, OpenGL, OpenGL ES, SDL, GLFW, X11, GDI+, WinAPI, Emscripten, Linux, Windows, Raspberry Pi}

\twentyitem{2016}{Tiled}{\href{https://github.com/bjorn/tiled/pull/1357}{https://github.com/bjorn/tiled/pull/1357}}{
A generic tile map editor. I added an ability to convert property type. Used technologies: C++, Qt, Linux, Windows.}
\end{twenty}



\section{Game Development Experience}

\begin{twenty} % Environment for a list with descriptions
	\twentyitem{2016-2017}{Wordlase}{\href{http://store.steampowered.com/app/602930}{http://store.steampowered.com/app/602930}}{
A word puzzle game, written in C. Game runs on Windows and Linux.

Used technologies: ansi C, libc, Nuklear, JSON, gzip, Python, SDL.

C code lines: $\sim$5k. Python code lines: $\sim$1k.

{\small Open source part - Nuklear+: \href{https://dexp.in/tools/nuklear-cross/}{https://dexp.in/tools/nuklear-cross/} }}


	\twentyitem{2015-2016}{Winter Novel}{\href{http://store.steampowered.com/app/485350}{http://store.steampowered.com/app/485350} }{
A visual novel ASCII game, written in low-level C. Windows, Linux.
		
A lot of functions were reimplemented. Used technologies: ansi C, no libc, WinAPI/SDL, OpenGL, Android. Strict Ansi C89.

Project size: $\sim$20k source code lines.}


%	\twentyitem{2014-2015}{One Manga Day}{\href{http://store.steampowered.com/app/365070}{http://store.steampowered.com/app/365070}}{
%A short visual novel originally written in Russian. The game is written in Python (RenPy engine). Music is self-made in Magix Music Maker. Graphics is made in Manga Maker Comipo.
%	
%{\small Source code: \href{https://github.com/DeXP/onemangaday}{https://github.com/DeXP/onemangaday}} }
\end{twenty}


\textit{Interests:} Linux, Games, Programming, Hardware, Fantasy books, Social dancing.\newline
\textit{Languages:} Russian (native), English (upper-intermediate), Czech (intermediate), Ukrainian (pre-intermediate), Belarusian (native).\newline
For more detailed information see my \href{https://www.linkedin.com/in/dexperix}{LinkedIn profile}.

%----------------------------------------------------------------------------------------
%	 INTERESTS
%----------------------------------------------------------------------------------------

%\section{Interests}

%Linux, Games, Programming, Hardware, Fantasy books, Martial arts.

%----------------------------------------------------------------------------------------
%	 Links
%----------------------------------------------------------------------------------------

\section{Links}

\begin{twentyshort}
	\twentyitemshort{LinkedIn:}{ \href{https://www.linkedin.com/in/dexperix}{https://www.linkedin.com/in/dexperix} }
	\twentyitemshort{GitHub:}{ \href{https://github.com/DeXP}{https://github.com/DeXP} }
	\twentyitemshort{Steam:}{ \href{https://store.steampowered.com/developer/dexp}{https://store.steampowered.com/developer/dexp} }
	\twentyitemshort{Google Play:}{ \href{https://play.google.com/store/apps/dev?id=7932817826050175353}{https://play.google.com/store/apps/dev?id=7932817826050175353} }
\end{twentyshort}


\end{document} 
